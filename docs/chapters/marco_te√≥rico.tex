\section{Estado del arte}
Un chatbot se puede definir como un programa diseñado para interactuar con los usuarios haciendo uso de lenguaje natural (ya sea por voz o texto) con la intención de hacer creer al usuario que se encuentra dialogando con una persona real \cite{designTechniques}. Para abordar esta problemática anteriormente se utilizaban algoritmos de \textit{pattern matching} para construir sistemas basados en reglas donde la interacción se limitaba a un simple patrón de pregunta-respuesta. \\

Sin embargo, hoy en dia con la aparición de nuevas tecnologías es posible crear sistemas mucho más complejos que nos permitan implementar patrones de conversación mucho más sofisticados, ofreciendo al usuario final una experiencia mucho más natural, y por lo tanto humana. En este nuevo campo la inteligencia artificial nos ha brindado una serie de herramientas muy útiles para para este objetivo: la comprensión del lenguaje natural, consciente del contexto, machine learning y  aprendizaje supervisado. \\

Explicar cada uno, referencias a papers etc


\section{Chatbots}
Como ya se ha mencionado, es de vital importancia que las conversaciones entre el \textit{bot} y el usuario no transmitan una sensación robótica, si no que sean fluidas y lo más humanas posibles. Para ello, un chatbot debe de integrar una serie de caracterísitcas mínimas que garantizen esta experiencia.


\subsection{Características básicas}
compresinón NLU, aprendizaje continuo ...

\subsection{Características sociales} 
Cómo debe el chatbot interactuar

\subsection{Otros aspectos} 


\subsection{Chatbots aplicados a la salud} 

\section{Herramientas para construir chatbots}
Análisis y comparativa de las distintas herramientas disponibles para construir chatbots

\section{Software existente similar}
¿Mover esta sección a estado del arte? 
