En el siguiente capítulo se desarrolla la propuesta de este trabajo de final de grado. Primeramente, se realiza un análisis de requisitos y se expone el diseño de la propuesta. Una vez la propuesta esta definida, se detalla todo el proceso de implementación seguido.

\section{Análisis de requisitos}
Como requisito fundamental por parte de los colaboradores de la Unidad de Enfermedades Infecciosas del Hospital General de Elche se propuso que el chatbot fuera capaz de responder a un mínimo de preguntas que bajo su criterio médico consideraron básicas para que el servicio contara con una base suficiente de conocimiento. Este listado fue entregado en una hoja de cálculo que se puede encontrar en los documentos anexos a este trabajo. El fichero consta de 131 preguntas con sus correspondientes respuestas y cubre las preguntas más frecuentes que suelen realizarse sobre el virus de la inmunodeficiencia humana (VIH). \\

Aunque inicialmente se partirá de esta base, se requiere que el sistema sea fácilmente escalable, de manera que a lo largo de un futuro desarrollo se pueda ir incorporando nueva información sin que suponga un problema. Por otra parte, una vez el \textit{bot} esté accesible al público y los usuarios interactúen con él se desea poder recopilar toda la información para posteriormente revisarla e incorporarla al \textit{bot}.\\

Junto a todos estos datos será también necesario incorporar todas aquellas expresiones auxiliares que suelen acompañar las conversaciones: afirmaciones, negaciones, saludos, despedidas, agradecimientos etc. Además, habrá que añadir un subconjunto de preguntas / respuestas básicas sobre el chatbot y otras cuestiones básicas que serán necesarias para dotar al asistente de una personalidad.\\

En conjunto se pretende que el chatbot se parezca lo máximo posible a una persona real, que demuestre estar versado en el tema sobre el que informa, que preste atención e interés en la conversación y que sea capaz de seguir la conversación. Todo esto es crucial a la hora de mantener una conversación con sentido y fluida que, en la medida de lo posible no haga sentir al usuario que está hablando con un robot. Desde el punto de vista técnico esto requerirá incorporar técnicas de comprensión del lenguaje natural (\textit{NLU}) y métodos de toma de decisiones en función del contexto. Por ende, es requisito fundamental que el \textit{bot} sea capaz de comprender las intenciones del usuario y responder en consecuencia.\\

Adicionalmente, y como parte de la aplicación se desea que exista un apartado donde los usuarios puedan consultar de manera más extensa información acerca del VIH. La presentación de esta información debe seguir un formato clásico y una estructura jerarquizada.\\

En lo referente a su distribución / publicación, al ser un servicio informativo y de concienciación se debe buscar la mejor solución para que sea fácilmente accesible. Por último, y no menos importante se debe de considerar también la problemática de la recogida de datos pues en una conversación especialmente sobre VIH se pueden introducir datos especialmente sensibles. Los usuarios deberán ser conscientes de como se van a tratar sus datos y aceptar los términos y condiciones.\\


Este análisis se presenta y acuerda en una reunión con los colaboradores la Unidad de Enfermedades Infecciosas del Hospital General de Elche para obtener su aprobación. Tanto las notas de la reunión como las diapositivas se pueden encontrar en los documentos anexos a este trabajo.\\

