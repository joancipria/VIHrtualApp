Después de analizar detenidamente los requisitos se procede a detallar la solución propuesta para el desarrollo de VIHrtual-App.

\subsection{Arquitectura de la aplicación}
\label{arquitectura}
Debido a los altos requerimientos computacionales y de espacio, \textit{Rasa} se basa en una arquitectura tradicional cliente-servidor. Para ello, pone a disposición de los desarrolladores un servidor ya integrado dentro del propio \textit{framework} con el cual es posible publicar el chatbot. Este \textit{backend} expone una \textit{API} conversacional que puede ser consumida a través de diferentes canales, como pueda ser una web o una plataforma de mensajería instantánea.\\

Teniendo en cuenta el carácter del servicio, se decide que el chatbot se ponga a disposición del público a través de una web de libre acceso y sin requerir registro. El objetivo es facilitar a los usuarios el acceso para que cualquier persona pueda acceder al servicio sin necesidad de introducir ningún dato personal ni instalar ninguna aplicación en su terminal. Por tanto, la web será el cliente (frontend) que ofrecerá la interfaz conversacional y establecerá comunicación con el servidor (\textit{backend}) de \textit{Rasa}.\\

Adicionalmente, y para algunas funciones más avanzadas, es necesario el uso de un segundo servidor denominado \textit{Rasa Action Server}. Este servicio permite la ejecución de funciones \textit{Python} y se utiliza para realizar tratamientos adicionales a los datos. En este caso, este servicio ha permitido implementar el mensajes de bienvenida que muestra el chatbot al inicio.\\

Tal y como se puede observar en la Figura \ref{fig:appStruct}, estos tres elementos (web, servidor conversacional \textit{Rasa} y \textit{Rasa Action Server}) conforman la arquitectura completa de la aplicación. La web establecerá comunicación mediante el protocolo \textit{HTTP} con los servidores de \textit{Rasa} y podrá ser publicada a través de un servidor web, o empaquetada en forma de aplicación móvil.

\begin{figure}[htbp]
\centering
\includegraphics[scale=0.2]{../images/app_structure.png} 
\caption{Esquema de la arquitectura de Vihrtual-App}
\label{fig:appStruct}
\end{figure}

%\subsection{Diseño de software}
%[TODO: Diagramas del diseño de software]


%\subsection{Diseño de diálogos}
%[TODO: Dialog flow charts etc]\\

\subsection{Mockups}
\label{mockups}
Tras el análisis de requisitos se empieza a trabajar en el diseño de la aplicación. Teniendo en cuenta que la web será accesible tanto por dispositivos móviles como por ordenadores se realizan dos diseños distintos: una versión móvil y otra de escritorio. Ambas versiones siguen los mismos principios pero tratan de adaptarse al tamaño de pantalla y hacer un uso adecuado del espacio.\\

Como fuente veraz de información el servicio debe transmitir fiabilidad y confianza, por esta razón se opta por un estilo sobrio y sencillo donde diferentes tonos de azul predominan \cite{colors}. Con ello se espera que transmita una sensación de paz y seguridad al usuario.\\

La aplicación se compone de tres pantallas distintas: selección del asistente, chat y consulta de información. En la Figura \ref{fig:mobile flow} podemos observar estas vistas junto con el diagrama de navegación de la aplicación. \\

\begin{figure}[htbp]
\centering
\includegraphics[scale=0.1]{../images/mobile_flow.png} 
\caption{Diagrama de navegación de la aplicación. Versión móvil}
\label{fig:mobile flow}
\end{figure}

\subsubsection{Selección del asistente}
En busca de establecer un primer vínculo entre el chatbot y el usuario se dota al sistema de dos personalidades distintas para humanizar la experiencia. En la primera pantalla a través de dos avatares (Juan y Elena) (ver Figura \ref{fig:mobile avatar} y \ref{fig:desktop avatar}) el usuario puede eligir con cual de estas dos identidades desea mantener una conversación y así empezar a crear la relación. Además de elegir con quién se siente más cómodo hablando, también lee el aviso sobre el tratamiento de sus datos. Tras ello, puede pulsar "\textit{Empezar}" para aceptar los términos y proceder al chat.\\

\begin{figure}[htbp]
\centering
\includegraphics[scale=0.2]{../images/mobile_avatar.png} 
\caption{Selección del asistente. Versión móvil}
\label{fig:mobile avatar}
\end{figure}

\begin{figure}[htbp]
\centering
\includegraphics[scale=0.2]{../images/desktop_avatar.png} 
\caption{Selección del asistente. Versión de escritorio}
\label{fig:desktop avatar}
\end{figure}

\subsubsection{Chat}
Desde un principio se tiene claro que el diseño debe girar alrededor de la interacción con el \textit{bot}, y por tanto, el chat debe ser el elemento principal de la aplicación. Por ello, en la versión móvil se opta por una disposición donde prácticamente todo el espacio es ocupado por la conversación (ver Figura \ref{fig:mobile chat}). En cambio, en la versión de escritorio el chat ocupa la mitad derecha de la pantalla, mostrando en la parte izquierda la imagen del avatar escogido con un pequeño texto informativo (ver Figura \ref{fig:desktop chat}). El estilo del chat sigue los patrones de diseño típicos y más o menos estandarizados de aplicaciones de mensajería instantánea como \textit{WhatsApp} o \textit{Telegram}. El motivo detrás de esta decisión es facilitar la usabilidad, ya que prácticamente la totalidad de los usuarios están familiarizados con este tipo de interfaces.\\

Tal y como se puede observar en las Figuras \ref{fig:mobile chat} y \ref{fig:desktop chat}, en la cabecera superior encontramos los elementos habituales: la imagen de perfil de con quién se está hablando y un pequeño texto que nos indica el estado: \textit{en línea} o \textit{escribiendo}. Ambos elementos juegan una papel importante en la simulación: mientras la conversación transcurre el usuario ve constantemente el avatar del asistente e inconscientemente realiza una asociación entre los mensajes y la imagen de la persona que esta viendo. Por otra parte, aunque el usuario sepa que es una simulación observar como el chatbot va cambiando su estado según esté escribiendo o no, vuelve a reforzar la idea debido al paralelismo con las aplicaciones anteriormente mencionadas.\\

Por último, a través del icono de la esquina superior derecha se accede a consultar la información adicional.\\

\begin{figure}[htbp]
\centering
\includegraphics[scale=0.2]{../images/mobile_chat.png} 
\caption{Chat. Versión móvil}
\label{fig:mobile chat}
\end{figure}

\begin{figure}[htbp]
\centering
\includegraphics[scale=0.2]{../images/desktop_chat.png} 
\caption{Chat. Versión de escritorio}
\label{fig:desktop chat}
\end{figure}


\subsubsection{Consulta de información}
La pantalla de información contiene una breve introducción al VIH y una serie de temas anidados que se pueden ir desplegando y consultando. Además, en la parte superior existe una barra de búsqueda que permite encontrar rápidamente ciertos términos o palabras clave. Si existen resultados que coincidan con la búsqueda se indica con un pequeño indicador amarillo sobre el desplegable y resaltando el término en cuestión (ver Figura \ref{fig:mobile search} y \ref{fig:desktop search}).\\

\begin{figure}[htbp]
\centering
\includegraphics[scale=0.2]{../images/mobile_search.png} 
\caption{Búsqueda de información. Versión móvil}
\label{fig:mobile search}
\end{figure}

\begin{figure}[htbp]
\centering
\includegraphics[scale=0.2]{../images/desktop_search.png} 
\caption{Búsqueda de información. Versión de escritorio}
\label{fig:desktop search}
\end{figure}

Estos \textit{mockups} se envían con un pequeño vídeo y encuesta a los colaboradores la Unidad de Enfermedades Infecciosas del Hospital General de Elche para obtener su \textit{feedback} y aprobación. Tanto sus indicaciones como la versión interactiva pueden encontrarse en los documentos anexos a este trabajo.\\