Durante la realización de este trabajo se han alcanzado con éxitos los objetivos planteados inicialmente. Se ha explorado el estado actual de creación de chatbots y se han estudiado distintos principios de diseño para aplicarlos al proyecto. Posteriormente, todo este conocimiento adquirido se ha plasmado en el diseño e implementación del asistente virtual, implementando todas las funcionalidades inicialmente previstas.\\

Tras su publicación, se han podido recopilar las conversaciones mantenidas por los usuarios con el servicio, de manera que ha sido posible trabajar en base a esta información para ir corrigiendo y mejorando el servicio. Este modo de trabajo ha producido una buena sinergia con los colaboradores de la Unidad de Enfermedades Infecciosas del Hospital General de Elche, que en última instancia, ha repercutido positivamente en el desarrollo. Gracias a la recogida de datos y su colaboración, ha sido posible recopilar y responder hasta 40 nuevas preguntas que inicialmente no estaban previstas, añadiendo una notable mejora a la base de conocimiento del chatbot.\\

Por otra parte, también se ha realizado una evaluación de usabilidad con 12 participantes que respondieron a los cuestionarios validados SUS y CUQ. Ambas pruebas arrojaron unos resultados de 85.625 y 80.342 sobre 100 respectivamente. Estas valoraciones positivas corroboran, en gran parte, muchas de las características sociales implementadas en el diseño de interacción. Por otro lado, también han servido para mostrar aquellos puntos donde es necesario continuar trabajando para mejorar la experiencia. En especial, el reconocimiento de entradas tiene margen de mejora y la cantidad de preguntas que puede responder el chatbot se debe aumentar.\\

Es importante considerar que, sólo 12 usuarios participaron en la prueba, y de ellos, la mayor parte fueron estudiantes. Por tanto, el análisis de los resultados debe realizarse teniendo en cuenta este contexto. Sería necesario un estudio más extenso donde, tanto la cantidad de encuestados como la variedad de sus perfiles fuera mayor. En este caso, gran parte de los participantes eran jóvenes y tenían estudios tecnológicos, por lo que los resultados obtenidos pueden estar algo sesgados.\\

En general, los resultados obtenidos han sido satisfactorios, y se espera que continuando con desarrollo basado en conversaciones se corrijan todos aquellos aspectos que son susceptibles de mejorar. En un futuro, la intención es que el chatbot sea también capaz de responder a otros temas relacionados con el VIH como puedan ser las las enfermedades de transmisión sexual.\\

A nivel personal, este trabajo de final de grado ha supuesto todo un impulso a la formación recibida en el grado. El reto de diseñar, dirigir e implementar una idea ha sido toda una experiencia. Durante todo el desarrollo ha sido necesario un aprendizaje permanente para poder resolver los problemas inicialmente planteados, superar las dificultades y lidiar con la planificación del proyecto. Igualmente, trabajar con un grupo de profesionales de un ámbito distinto ha supuesto una mejora de mis capacidades comunicativas, y sobretodo, una experiencia muy enriquecedora. Este trabajo ha supuesto para mi toda una introducción en el mundo del desarrollo de chatbots, y llevar a cabo un proyecto como este me ha otorgado las aptitudes necesarias para orientar mi futura vida profesional hacia el sector de la inteligencia artificial.\\
