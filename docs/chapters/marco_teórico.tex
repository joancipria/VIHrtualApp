\section{Estado del arte}
%El habla es una de las formas de comunicación más complejas y poderosas que dispone el ser humano. Esta comprende el uso lenguaje y otros signos como del meta-lenguaje nos permite entablar conversaciones e intercambiar información en un entendimiento mutuo entre 2 o más personas.

%En los últimos años con la aparición de nuevas tecnologías la interacción entre humanos y ordenadores se ha estado moviendo cada vez más hacia sistemas basados en voz o texto, popularizándose asistentes virtuales como Siri, Alexa, Google Assistant, Cortana etc. Esto se debe a que el lenguaje natural se está revelando como una manera muy competente para construir interfaces personalizadas, permitiendo que los usuarios puedan comunicarse con sus móviles/ordenadores utilizando sus propias palabras \cite{designTechniques}.\\

Como se ha visto en el capítulo anterior, con la aparición de nuevas tecnologías la interacción entre humano y ordenadores ha establecido nuevo escenario: las clásicas interfaces gráficas genéricas dejan paso a un nuevo paradigma de interacción mucho más intuitivo dónde no es necesario hacer clicks ni teclear datos para realizar ciertas operaciones \cite{conversationSystems}. Ciertos procesos que antes podían resultar algo tediosos como reservar un vuelo ahora ya son posible con un simple "\textit{Google, resérvame un vuelo a París}", sin necesidad de navegar a través distintas webs ni rellenar formularios.\\

Por tanto, un chatbot o asistente virtual se puede definir como una interfaz (programa) diseñado para que los usuarios puedan interactuar con sus dispositivos haciendo uso de lenguaje natural, ya sea directamente por voz o por texto. El objetivo principal es siempre transmitir la sensación al usuario que se encuentra dialogando con una persona real y mantener esta pequeña ilusión el mayor tiempo posible. Para ello es crucial que los chatbots sean capaces de demostrar entendimiento y de resolver los problemas que los usuarios planteen.\\


Para abordar esta problemática la principal estrategia es reconocer que intenciones tiene el usuario al expresarse. Para ello, anteriormente se utilizaban simplemente algoritmos de \textit{pattern matching} para construir sistemas basados en reglas donde la interacción se limitaba a un simple patrón de pregunta-respuesta. Sin embargo, hoy en día con la aparición de nuevas tecnologías es posible crear sistemas más complejos que nos permitan implementar patrones de conversación mucho más sofisticados, ofreciendo al usuario final una experiencia mucho más natural, y por lo tanto humana. En este nuevo campo la inteligencia artificial nos ha brindado una serie de herramientas muy útiles para para este objetivo: la comprensión del lenguaje natural, consciente del contexto, machine learning y  aprendizaje supervisado. \\

Explicar cada uno, referencias a papers etc


\section{Chatbots}
Como ya se ha mencionado, es de vital importancia que las conversaciones entre el \textit{bot} y el usuario no transmitan una sensación robótica, si no que sean fluidas y lo más humanas posibles. Para ello, un chatbot debe de integrar una serie de caracterísitcas mínimas que garantizen esta experiencia.


\subsection{Características básicas}
compresinón NLU, aprendizaje continuo ...

\subsection{Características sociales} 
Cómo debe el chatbot interactuar

\subsection{Otros aspectos} 


\subsection{Chatbots aplicados a la salud} 

\section{Herramientas para construir chatbots}
Análisis y comparativa de las distintas herramientas disponibles para construir chatbots

\section{Software existente similar}
¿Mover esta sección a estado del arte? 
