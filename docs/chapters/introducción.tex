\section{Introducción}
La inteligencia artificial (IA) está transformando el mundo actual y la manera en la cual nos comunicamos con las máquinas, popularizándose cada vez más los asistentes y sistemas basados en voz. Esto es debido a que el lenguaje natural se está revelando como una forma muy óptima y competente de crear interfaces personalizadas que permitan a cada usuario interactuar utilizando sus propias palabras \cite{naturalDialogue} . Ante este nuevo escenario empezamos a ver como se popularizan los chatbots: programas diseñados para interactuar con los usuarios utilizando lenguaje natural (por voz o texto) con el objetivo de hacer creer al usuario está hablando con una persona real. \\

En el ámbito sanitario esto no ha sido una excepción y se siguen estos avances con gran interés en busca de asistentes virtuales que permitan mejorar y automatizar ciertos procesos médicos \cite{healthAgents}. No sólo se pretende obtener un beneficio inmediato al resolver las dudas más comunes y automatizar ciertas consultas, si no que a largo plazo esta alfabetización y acceso libre a la información puede contribuir a una sociedad más formada y consciente que mejore de manera activa su salud.\\ 

Por otro lado, en España, el virus de la inmunodeficiencia humana (VIH) continua representando un problema. Según la última estimación publicada por el Plan Nacional de Sida, en España viven un total de 151.387 personas con VIH. Sólo en el año 2019 se notificaron 2.698 nuevos diagnósticos de VIH, de los cuales el 45.9\% presentaron diagnóstico tardío \cite{vihEspana}. Bajo este contexto surge este trabajo de final de grado. Un proyecto que busca ayudar a detener la transmisión e informar ofreciendo un servicio al público donde poder obtener información veraz y relevante. Un lugar dónde todas aquellas personas puedan resolver las dudas más comunes sobre el VIH y aprendan a tomar medidas de prevención. En definitiva, facilitar la mayor información posible a los usuarios para que estén protegidos y sean conscientes de la problemática actual. En este trabajo de final de grado se lleva a cabo el desarrollo de esta idea diseñando e implementando un chatbot completamente funcional que es accesible vía web.\\

Este proyecto está enmarcado dentro de VIHrtual-App, un proyecto de investigación con la colaboración de la Universitat Politècnica de València (UPV), la Fundación FISABIO y la Unidad de Enfermedades Infecciosas del Hospital General de Elche. La participación por parte de los colaboradores del hospital ha sido clave en el asesoramiento médico y veracidad de la información.\\ 

El presente documento describe todo el proceso que se ha seguido durante todo el desarrollo. Desde el estudio previo hasta su despliegue y validación.\\

\section{Objetivos del proyecto}
El objetivo principal de este proyecto es el desarrollo de un asistente virtual para la prevención del VIH. Los usuarios podrán interactuar con él mediante una interfaz conversacional y resolver sus dudas. Para llevar a cabo esta tarea es necesario cumplir una serie de objetivos secundarios:\\

\begin{itemize}
	\item Revisar el estado del arte sobre la creación de chatbots, obteniendo información sobre las técnicas actuales de diseño e implementación.
	\item Estudiar las características sociales en el diseño de la interacción entre humanos y chatbots.
	\item Analizar implementaciones similares de chatbots aplicados al campo de la salud.
	\item Detectar los requisitos del proyecto y analizar distintas herramientas para el desarrollo del asistente.
	\item Diseñar e implementar el servicio.
	\item Validar el proyecto con usuarios reales.
\end{itemize}

\section{Estructura del documento}
\noindent \textbf{Capítulo I: Introducción}\\
En el primer capítulo se realiza una breve introducción al proyecto, cuales son sus objetivos y la motivación que ha empujado a su desarrollo. Asimismo, también se describe la metodología de trabajo empleada durante toda su realización y una breve descripción de la estructura del documento.  \\

\noindent \textbf{Capítulo II: Marco Teórico}\\
En este capítulo se pretende dar a conocer la base teórica sobre las cual se basa el proyecto: se describe el estado actual del arte, indaga en los aspectos a considerar en el diseño del chatbot y se analizan las distintas herramientas disponibles para la implementación.\\

\noindent \textbf{Capítulo III: Desarrollo de la propuesta}\\
En el tercer capítulo se muestra todo el proceso del desarrollo de la propuesta, desde el análisis de requisitos hasta el diseño e implementación del sistema. \\

\noindent \textbf{Capítulo IV: Pruebas de validación}\\
En este capítulo se realiza una evaluación final del sistema para comprobar que todos los requisitos iniciales planteados hayan sido satisfechos.\\

\noindent \textbf{Capítulo V: Conclusiones}\\
En sexto capítulo se realiza una valoración final del proyecto teniendo en cuenta las pruebas de validación.\\

\noindent \textbf{Capítulo VI: Referencias}\\
En este último capítulo se listan todas las referencias consultadas y citadas durante el desarrollo de este trabajo.\\