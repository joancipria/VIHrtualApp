\section{Introducción}
La inteligencia artificial (IA) está transformando el mundo actual y la manera en la cual nos comunicamos con las máquinas, popularizándose cada vez más los asistentes y sistemas basados en voz. Esto es debido a que el lenguaje natural se está revelando como una forma muy óptima y competente de crear interfaces personalizadas que permitan a cada usuario interactuar utilizando sus propias palabras \cite{naturalDialogue} . Ante este nuevo escenario empezamos a ver como se popularizan los chatbots: programas diseñados para interactuar con los usuarios utilizando lenguaje natural (por voz o texto) con el objetivo de hacer creer al usuario está hablando con una persona real. \\

En el ámbito sanitario esto no ha sido una excepción y se siguen estos avances con gran interés en busca de asistentes virtuales que permitan mejorar y automatizar ciertos procesos médicos\cite{healthAgents}. No sólo se pretende obtener un beneficio inmediato al resolver las dudas más comunes y automatizar ciertas consultas, si no que a largo plazo esta alfabetización y acceso libre a la información puede contribuir a una sociedad más formada y consciente que mejore de manera activa su salud. En España por ejemplo, el virus de la inmunodeficiencia humana (VIH) continua representando un problema ... [TODO: Situación y problemática actual VIH en España]\\

Bajo este contexto se crea Vihrtual-App, un proyecto de investigación con la colaboración de la Universitat Politècnica de València, la Fundación FISABIO y la Unidad de Enfermedades Infecciosas del Hospital General de Elche. El proyecto consiste en el diseño e implementación de un chatbot que ofreceza al público un servicio para obtener información veraz y relevante sobre el VIH y dónde puedan resolver las dudas más comunes. El objetivo es ofrecer una herramienta más que ayude a detener la transmisión del virus ofreciendo la mayor información posible a los usuarios para que estén protegidos y sean conscientes de la problemática actual.\\ % así como de otras enfermedades de transmisión sexual.\\

En este trabajo de final de grado se lleva a cabo el desarrollo de esta idea diseñando e implementando un chatbot completamente funcional que es accesible tanto vía web como desde una \textit{app} móvil. Asimismo, el presente documento intenta describir todo el proceso que se ha seguido desde el estudio previo hasta su despligue.


%\section{Motivación}
%Motivación: investigación, un tema molt interessant i una oportunitat per fer un bon ús de la tecnologia millorant la vida de les persones

\section{Objetivos del proyecto}
[TODO: Objetivos del proyecto]

\section{Metodología}
\input{chapters/introduccion/metodología}

\section{Estructura del documento}
\noindent \textbf{Capítulo I: Introducción}\\
En el primer capítulo se realiza una breve introducción al proyecto, cuales son sus objetivos y la motivación que ha empujado a su desarrollo. Asimismo, también se describe la metodología de trabajo empleada durante toda su realización y una breve descripción de la estructura del documento.  \\

\noindent \textbf{Capítulo II: Marco Teórico}\\
En este capítulo se pretende dar a conocer la base teórica sobre las cual se basa el proyecto: se describe el estado actual del arte, indaga en los aspectos a considerar en el diseño del chatbot y se analizan las distintas herramientas disponibles para la implementación.\\

\noindent \textbf{Capítulo III: Desarrollo de la propuesta}\\
En el tercer capítulo se muestra todo el proceso del desarrollo de la propuesta, desde el análisis de requisitos hasta el diseño e implementación del sistema. \\

\noindent \textbf{Capítulo IV: Pruebas de validación}\\
En este capítulo se realiza una evaluación final del sistema para comprobar que todos los requisitos iniciales planteados hayan sido satisfechos.\\

\noindent \textbf{Capítulo V: Conclusiones}\\
En sexto capítulo se realiza una valoración final del proyecto teniendo en cuenta las pruebas de validación.\\

\noindent \textbf{Capítulo VI: Bibliografía}\\
En este último capítulo se listan todas las referencias consultadas y citadas durante el desarrollo de este trabajo.\\