\documentclass[11pt]{book}
\usepackage[utf8]{inputenc}
\usepackage{graphicx}
\graphicspath{ {images/} }
%\usepackage{fancyhdr}
%\pagestyle{fancy}
\usepackage[spanish]{babel}
\usepackage{geometry}

%----------------------------------------------------------------------------------------
%	MARGIN SETTINGS
%----------------------------------------------------------------------------------------

\geometry{
	paper=a4paper, % Change to letterpaper for US letter
	inner=2.5cm, % Inner margin
	outer=3.8cm, % Outer margin
	bindingoffset=.5cm, % Binding offset
	top=1.5cm, % Top margin
	bottom=1.5cm, % Bottom margin
	%showframe, % Uncomment to show how the type block is set on the page
}


\title{
{TFG Vihrtual-App}\\
{\large Universitat Politècnica de València}\\
%{\includegraphics{university.jpg}}
}
\author{Joan Ciprià Moreno Teodoro}
\date{Gandía,2021}

\begin{document}

\maketitle

\chapter*{Resumen}
La inteligencia artificial (AI) está transformando el mundo actual y la manera en la cual nos comunicamos con las máquinas, popularizándose cada vez más los sistemas basados en voz. Esto es debido a que el lenguaje natural se está revelando como una forma muy óptima y competente para crear interfaces personalizadas que permitan a cada usuario interactuar con sus propias palabras. \\

Ante este nuevo escenario empezamos a ver como se popularizan los chatbots: programas diseñados para interactuar con los usuarios utilizando un lenguaje natural (por voz o texto) con el objetivo de hacer creer al usuario está hablando con una persona real. \\

En el ámbito sanitario esto no ha sido una excepción y se siguen estas tecnologías con gran interés en busca de sistemas que permitan automatizar ciertos procesos médicos. \\

Bajo este contexto surge Vihrtual-App, un chatbot creado con la colaboración del Hospital universitario de Elche y la Universitat Politència de València con la intención de concienciar y resolver a la población dudas sobre el VIH y otras enfermedades de transmisión sexual.
\\
\\
\textbf{Palabras clave:} Inteligencia artificial, lenguaje natural, chatbots, salud, virus de la inmunodeficiencia humana

\chapter*{Abstract}
To mum and dad

\tableofcontents

\chapter{Introducción}
\section{Introducción}
La inteligencia artificial (IA) está transformando el mundo actual y la manera en la cual nos comunicamos con las máquinas, popularizándose cada vez más los asistentes y sistemas basados en voz. Esto es debido a que el lenguaje natural se está revelando como una forma muy óptima y competente para crear interfaces personalizadas que permitan a cada usuario interactuar utilizando sus propias palabras \cite{naturalDialogue} . Ante este nuevo escenario empezamos a ver como se popularizan los chatbots: programas diseñados para interactuar con los usuarios utilizando lenguaje natural (por voz o texto) con el objetivo de hacer creer al usuario está hablando con una persona real. \\

En el ámbito sanitario esto no ha sido una excepción y se siguen estos avances con gran interés en busca de asistentes virtuales que permitan mejorar y automatizar ciertos procesos médicos\cite{healthAgents}. Por ejemplo, un asistente virtual en este contexto podría resultar útil para ofrecer información y concienciar a la población general sobre una determinada enfermedad. Por otra parte, también podría resolver dudas específicas a pacientes y familiares sobre un análisis clínico \cite{healthAgents}, aclarando conceptos médicos y aconsejando mejores hábitos de vida. En todos estos casos, no sólo se puede obtener un beneficio directo al resolver las dudas más comunes y automatizar ciertas consultas, si no que a largo plazo esta alfabetización y acceso libre a la información puede contribuir a una sociedad más formada y consciente que mejore su salud.\\

Por otra parte, en España el virus de la inmunodeficiencia humana (VIH) continua representando un problema ...\\

Bajo este contexto se crea Vihrtual-App, un proyecto de investigación con la colaboración de la Universitat Politècnica de València, la Fundación FISABIO y la Unidad de Enfermedades Infecciosas del Hospital General de Elche. El proyecto consiste en el diseño e implementación de un chatbot que ofreceza al público un servicio donde poder obtener información veraz y relevante sobre el VIH. Un lugar dónde puedan resolver sus dudas más comunes y aprender más sobre el tema. Por tanto, el objetivo final es ofrecer una herramienta más que ayude a detener la transmisión del virus dando a conocer la problemática y el riesgo actual del VIH.\\ % así como de otras enfermedades de transmisión sexual.\\

En este trabajo de final de grado se lleva a cabo el desarrollo de esta idea diseñando e implementando un chatbot completamente funcional que es accesible tanto vía web como desde una \textit{app} móvil. Asimismo, el presente documento intenta describir todo el proceso que se ha seguido.


%\section{Motivación}
%Motivación: investigación, un tema molt interessant i una oportunitat per fer un bon ús de la tecnologia millorant la vida de les persones

\section{Objetivos del proyecto}
Objetivos del proyecto

\section{Metodología}
Durante todo el desarrollo se ha empleado la metodología ágil \textit{Scrum}. Este método se caracteriza porque se realizan entregas parciales y regulares del producto final priorizando aquellas tareas que sean de mayor importancia para el resultado deseado. El desarrollo se ejecuta en ciclos temporales cortos y de duración fija (\textit{Sprints}) y al finalizar deben proporcionar un incremento del producto que sea susceptible de ser entregado. Tras esto, se realiza una reunión con el \textit{Product Owner} para evaluar el resultado \textit{Sprint review} y se planifica la siguiente iteración (\textit{Sprint planning}).\\

\begin{figure}[h]
\centering
\includegraphics[scale=0.4]{../images/scrum.png} 
\caption{Metodología Scrum}
\label{fig:x scrum}
\end{figure}

Dada la naturaleza del proyecto y la poca experiencia previa en el desarrollo de chatbots esta forma de trabajar aporta la suficiente flexibilidad para subsanar los posibles errores o problemas que surjan durante el proceso de diseño o desarrollo. Por otra lado, una parte crucial en la construcción de cualquier chatbot es poder empezar a probar el servicio con usuarios reales cuando antes, de manera que se pueda ir incorporando a la base de datos toda la información recopilada y el desarrollo se pueda beneficiar de este \textit{feedback}. Es en este punto donde la entrega de valor en períodos cortos de tiempo de \textit{Scrum} es idónea para el proyecto, ya que estas iteraciones posibilitan tener una producto mínimo viable (\textit{PMI}) cuanto antes.\\

En este proyecto los \textit{Sprints} han sido de 2 semanas naturales de duración, realizando el día 14 una reunión con el \textit{Product Owner} (los tutores del proyecto) para valorar el progreso realizado y determinar las próximos pasos a seguir. \\

Por otra parte, también se ha mantenido un contacto constante a través de reuniones \textit{online} y correos electrónicos con los investigadores del Hospital General de Elche para contar en todo momento con su visto bueno en temas médicos y de diseño.\\

En los documentos anexos a este trabajo se pueden encontrar las actas de todas estas reuniones con la fecha, los participantes y los temas tratados.\\


\section{Roadmap}
Siguiendo la metodología \textit{Scrum} se propone una ruta de desarrollo (ver figura \ref{fig:roadmap desarrollo}) centrada en la obtención de un producto mínimo viable lo antes posible. Para ello primeramente se realizan los diseños del chatbot, se organiza la información recibida y se trabaja para la obtención de un \textit{PMI} en versión web. A partir de ese punto es se publica una versión de prueba del \textit{bot} para empezar con la obtención de datos reales  y realizar distintas iteraciones sobre el producto hasta conseguir el resultado deseado. Una vez se alcanza el nivel de madurez requerido, la web se empaqueta como \textit{app} para móvil y se publica.\\

\begin{figure}[htbp]
\centering
\includegraphics[scale=0.4]{../images/roadmap.png} 
\caption{\textit{Roadmap} propuesto para el desarrollo de Vihrtual-App}
\label{fig:roadmap desarrollo}
\end{figure}




\section{Estructura del documento}
\noindent \textbf{Capítulo I: Introducción}\\
En el primer capítulo se realiza una breve introducción al proyecto, cuales son sus objetivos y la motivación que ha empujado a su desarrollo. Asimismo, también se describe la metodología de trabajo empleada durante toda su realización y una breve descripción de la estructura del documento.  \\

\noindent \textbf{Capítulo II: Marco Teórico}\\
En este capítulo se pretende dar a conocer la base teórica sobre las cual se basa el proyecto: se describe el estado actual del artes, indaga en los aspectos a considerar en el diseño del chatbot y se analizan las distintas herramientas disponibles la implementación.\\

\noindent \textbf{Capítulo III: Desarrollo de la propuesta}\\
En el tercer capítulo se muestra todo el proceso del desarrollo de la propuesta, desde el análisis de requisitos hasta el diseño e implementación del sistema. \\

\noindent \textbf{Capítulo IV: Pruebas de validación}\\
En este capítulo se realiza una evaluación final del sistema para comprobar que todos los requisitos iniciales planteados hayan sido satisfechos.\\

\noindent \textbf{Capítulo V: Conclusiones}\\
En sexto capítulo se realiza una valoración final del proyecto teniendo en cuenta las pruebas de validación.

\noindent \textbf{Capítulo VI: Conclusiones}\\
En este último capítulo se listan todas las referencias consultadas y citadas durante el desarrollo de este trabajo. 

\chapter{Marco teórico}
\section{Interacción Hombre-Máquina y lenguaje}
El lenguaje es una de las formas más naturales y poderosas de comunicación entre humanos. Es complejo, incluye el metalenguaje, va más allá del entendimiento litaral de las palabras, ironias, sonidos, tonos, contextos ... etc ... \\

Con la aparición de nuevas tecnologías y la creciente capacidad de computación muchos se  han lanzado a la investigación / creación de técnicas para el reconocimiento del lenguaje para utilizarlo como interaccion entre hombre -máquinas ...\\

Estos avances establecen las bases del futuro de la interacción entre los usuarios y sus dispositivos a la vez que establecen un nuevo escenario donde ya no será necesario utilizar ratón y teclado. Las clásicas interfaces genéricas dejan paso a un nuevo paradigma de interacción mucho más intuitivo dónde no es necesario hacer clicks ni teclear datos para realizar ciertas operaciones \cite{conversationSystems}. Aunque el reconocimiento todavía se encuentra en una fase temprana y está lejos de poder interpretar conversaciones complejas y mantener cierta coherencia,realidad ciertos procesos que antes podían resultar algo tediosos como reservar un vuelo ahora ya son posible con un simple "\textit{Google, resérvame un vuelo a París}", sin necesidad de navegar a través distintas webs ni rellenar formularios.\\

\section{Chatbots}
Un chatbot o asistente virtual se puede definir como una interfaz (programa) diseñado para que los usuarios puedan interactuar con sus dispositivos haciendo uso de lenguaje natural, ya sea directamente por voz o por texto. El objetivo principal es siempre transmitir la sensación al usuario que se encuentra dialogando con una persona real y mantener esta ilusión el mayor tiempo posible. Para ello es crucial que los chatbots sean capaces de demostrar entendimiento y de resolver los problemas que los usuarios planteen.\\

\subsection{Clasificación}
[TODO: Tipos de chatbots, task-oriented etc]

\subsection{Técnicas de diseño }
La principal estrategia es reconocer que intenciones tiene el usuario al expresarse. Para ello, anteriormente se utilizaban simplemente algoritmos de \textit{pattern matching} para construir sistemas basados en reglas donde la interacción se limitaba a un simple patrón de pregunta-respuesta. Sin embargo, hoy en día con la aparición de nuevas tecnologías es posible crear sistemas más complejos que nos permitan implementar patrones de conversación mucho más sofisticados, ofreciendo al usuario final una experiencia mucho más natural, y por lo tanto humana. En este nuevo campo la inteligencia artificial nos ha brindado una serie de herramientas muy útiles para para este objetivo: la comprensión del lenguaje natural, consciente del contexto, machine learning y  aprendizaje supervisado. \\

\subsection{Interacción y características sociales}
Como ya se ha mencionado, es de vital importancia que las conversaciones entre el \textit{bot} y el usuario no transmitan una sensación robótica, si no que sean fluidas y lo más humanas posibles. Para ello, un chatbot debe de integrar una serie de caracterísitcas mínimas que garantizen esta experiencia.\\

- Cracteristics básicas y sociales, como debe interactuar

\section{Estado del arte}
Chatbots en la salud, tipos, agents in health, IRA, wakamola etc\\


\section{Herramientas para construir chatbots}
Análisis y comparativa de las distintas herramientas disponibles para construir chatbots



\chapter{Desarrollo de la Propuesta}

\section{Requisitos}
Como requisito fundamental por parte de los colaboradores de la Unidad de Enfermedades Infecciosas del Hospital General de Elche se propuso que el chatbot fuera capaz de responder a un mínimo de preguntas que bajo su criterio médico consideraron básicas para que el servicio contara con una base suficiente de conocimiento. Este listado fue entregado en una hoja de cálculo que se puede encontrar en los documentos anexos a este trabajo. El fichero consta de 131 preguntas con sus correspondientes respuestas y cubre las preguntas más frecuentes que suelen realizarse sobre el virus de la inmunodeficiencia humana (VIH). \\

Aunque inicialmente se partirá de esta base se requiere que el sistema sea fácilmente escalable de manera que a lo largo de un futuro desarrollo se pueda ir incorporando nueva información sin que suponga un problema. La intención en un futuro es que el chatbot sea también capaz de responder a otros temas relacionados como las enfermedades de transmisión sexual, por tanto este es un factor a considerar para evitar problemas posteriores. Por otra parte, una vez el \textit{bot} esté accesible al público y los usuarios interactúen con él se desea poder recopilar toda la información recogida para posteriormente revisarla e incorporarla al \textit{bot}.\\

Junto a todos estos datos será también necesario incorporar todas aquellas expresiones auxiliares que suelen acompañar las conversaciones: afirmaciones, negaciones, saludos, despedidas, agradecimientos etc. Además, habrá que añadir un subconjunto de preguntas / respuestas básicas sobre el chatbot y otras cuestiones básicas que serán necesarias para dotar al asistente de una personalidad.\\

En conjunto se pretende que el chatbot se parezca lo máximo posible a un apersona real, demuestre que dominar el tema sobre el que informa, que muestre atención e interés en la conversación y que sea capaz de seguir el hilo. Todo esto es crucial a la hora de mantener una conversación con sentido y fluida que, en la medida de lo posible no haga sentir al usuario que está hablando con un robot. Desde el punto de vista técnico esto requerirá incorporar técnicas de comprensión del lenguaje natural (\textit{NLU}) y ser ser capaz de tomar decisiones en función del contexto. Por ende, es requisito fundamental que el \textit{bot} sea capaz de comprender las intenciones del usuario y responder en consecuencia.\\

Adicionalmente, y como parte de la aplicación se desea que exista un apartado donde los usuarios puedan consultar de manera más extensa información acerca del VIH. La presentación de esta información debe seguir un formato clásico y una estructura jerarquizada.\\

En lo referente a su distribución / publicación, al ser un servicio informativo y de concienciación se debe buscar la mejor solución para que sea fácilmente accesible. Por último, y no menos importante se debe de considerar también la problemática de la recogida de datos pues en una conversación especialmente sobre VIH se pueden introducir datos especialmente sensibles. Los usuarios deberán ser conscientes de como se van a tratar sus datos y aceptar los términos y condiciones.\\




\section{Metodología}
Durante todo el desarrollo se ha empleado la metodología ágil \textit{Scrum}. Este método se caracteriza porque se realizan entregas parciales y regulares del producto final priorizando aquellas tareas que sean de mayor importancia para el resultado deseado. El desarrollo se ejecuta en ciclos temporales cortos y de duración fija (\textit{Sprints}) y al finalizar deben proporcionar un incremento del producto que sea susceptible de ser entregado. Tras esto, se realiza una reunión con el \textit{Product Owner} para evaluar el resultado \textit{Sprint review} y se planifica la siguiente iteración (\textit{Sprint planning}).\\

\begin{figure}[h]
\centering
\includegraphics[scale=0.4]{../images/scrum.png} 
\caption{Metodología Scrum}
\label{fig:x scrum}
\end{figure}

Dada la naturaleza del proyecto y la poca experiencia previa en el desarrollo de chatbots esta forma de trabajar aporta la suficiente flexibilidad para subsanar los posibles errores o problemas que surjan durante el proceso de diseño o desarrollo. Por otra lado, una parte crucial en la construcción de cualquier chatbot es poder empezar a probar el servicio con usuarios reales cuando antes, de manera que se pueda ir incorporando a la base de datos toda la información recopilada y el desarrollo se pueda beneficiar de este \textit{feedback}. Es en este punto donde la entrega de valor en períodos cortos de tiempo de \textit{Scrum} es idónea para el proyecto, ya que estas iteraciones posibilitan tener una producto mínimo viable (\textit{PMI}) cuanto antes.\\

En este proyecto los \textit{Sprints} han sido de 2 semanas naturales de duración, realizando el día 14 una reunión con el \textit{Product Owner} (los tutores del proyecto) para valorar el progreso realizado y determinar las próximos pasos a seguir. \\

Por otra parte, también se ha mantenido un contacto constante a través de reuniones \textit{online} y correos electrónicos con los investigadores del Hospital General de Elche para contar en todo momento con su visto bueno en temas médicos y de diseño.\\

En los documentos anexos a este trabajo se pueden encontrar las actas de todas estas reuniones con la fecha, los participantes y los temas tratados.\\


\section{Roadmap}
Siguiendo la metodología \textit{Scrum} se propone una ruta de desarrollo (ver figura \ref{fig:roadmap desarrollo}) centrada en la obtención de un producto mínimo viable lo antes posible. Para ello primeramente se realizan los diseños del chatbot, se organiza la información recibida y se trabaja para la obtención de un \textit{PMI} en versión web. A partir de ese punto es se publica una versión de prueba del \textit{bot} para empezar con la obtención de datos reales  y realizar distintas iteraciones sobre el producto hasta conseguir el resultado deseado. Una vez se alcanza el nivel de madurez requerido, la web se empaqueta como \textit{app} para móvil y se publica.\\

\begin{figure}[htbp]
\centering
\includegraphics[scale=0.4]{../images/roadmap.png} 
\caption{\textit{Roadmap} propuesto para el desarrollo de Vihrtual-App}
\label{fig:roadmap desarrollo}
\end{figure}




\section{Diseño}
Después de analizar detenidamente los requisitos se procede a detallar la solución propuesta para el desarrollo de VIHrtual-App.

\subsection{Arquitectura de la aplicación}
\label{arquitectura}
Debido a los altos requerimientos computacionales y de espacio, \textit{Rasa} se basa en una arquitectura tradicional cliente-servidor. Para ello, pone a disposición de los desarrolladores un servidor ya integrado dentro del propio \textit{framework} con el cual es posible publicar el chatbot. Este \textit{backend} expone una \textit{API} conversacional que puede ser consumida a través de diferentes canales, como pueda ser una web o una plataforma de mensajería instantánea.\\

Teniendo en cuenta el carácter del servicio, se decide que el chatbot se ponga a disposición del público a través de una web de libre acceso y sin requerir registro. El objetivo es facilitar a los usuarios el acceso para que cualquier persona pueda acceder al servicio sin necesidad de introducir ningún dato personal ni instalar ninguna aplicación en su terminal. Por tanto, la web será el cliente (frontend) que ofrecerá la interfaz conversacional y establecerá comunicación con el servidor (\textit{backend}) de \textit{Rasa}.\\

Adicionalmente, y para algunas funciones más avanzadas, es necesario el uso de un segundo servidor denominado \textit{Rasa Action Server}. Este servicio permite la ejecución de funciones \textit{Python} y se utiliza para realizar tratamientos adicionales a los datos. En este caso, este servicio ha permitido implementar el mensajes de bienvenida que muestra el chatbot al inicio.\\

Tal y como se puede observar en la Figura \ref{fig:appStruct}, estos tres elementos (web, servidor conversacional \textit{Rasa} y \textit{Rasa Action Server}) conforman la arquitectura completa de la aplicación. La web establecerá comunicación mediante el protocolo \textit{HTTP} con los servidores de \textit{Rasa} y podrá ser publicada a través de un servidor web, o empaquetada en forma de aplicación móvil.

\begin{figure}[htbp]
\centering
\includegraphics[scale=0.2]{../images/app_structure.png} 
\caption{Esquema de la arquitectura de Vihrtual-App}
\label{fig:appStruct}
\end{figure}

%\subsection{Diseño de software}
%[TODO: Diagramas del diseño de software]


%\subsection{Diseño de diálogos}
%[TODO: Dialog flow charts etc]\\

\subsection{Mockups}
\label{mockups}
Tras el análisis de requisitos se empieza a trabajar en el diseño de la aplicación. Teniendo en cuenta que la web será accesible tanto por dispositivos móviles como por ordenadores se realizan dos diseños distintos: una versión móvil y otra de escritorio. Ambas versiones siguen los mismos principios pero tratan de adaptarse al tamaño de pantalla y hacer un uso adecuado del espacio.\\

Como fuente veraz de información el servicio debe transmitir fiabilidad y confianza, por esta razón se opta por un estilo sobrio y sencillo donde diferentes tonos de azul predominan \cite{colors}. Con ello se espera que transmita una sensación de paz y seguridad al usuario.\\

La aplicación se compone de tres pantallas distintas: selección del asistente, chat y consulta de información. En la Figura \ref{fig:mobile flow} podemos observar estas vistas junto con el diagrama de navegación de la aplicación. \\

\begin{figure}[htbp]
\centering
\includegraphics[scale=0.1]{../images/mobile_flow.png} 
\caption{Diagrama de navegación de la aplicación. Versión móvil}
\label{fig:mobile flow}
\end{figure}

\subsubsection{Selección del asistente}
En busca de establecer un primer vínculo entre el chatbot y el usuario se dota al sistema de dos personalidades distintas para humanizar la experiencia. En la primera pantalla a través de dos avatares (Juan y Elena) (ver Figura \ref{fig:mobile avatar} y \ref{fig:desktop avatar}) el usuario puede eligir con cual de estas dos identidades desea mantener una conversación y así empezar a crear la relación. Además de elegir con quién se siente más cómodo hablando, también lee el aviso sobre el tratamiento de sus datos. Tras ello, puede pulsar "\textit{Empezar}" para aceptar los términos y proceder al chat.\\

\begin{figure}[htbp]
\centering
\includegraphics[scale=0.2]{../images/mobile_avatar.png} 
\caption{Selección del asistente. Versión móvil}
\label{fig:mobile avatar}
\end{figure}

\begin{figure}[htbp]
\centering
\includegraphics[scale=0.2]{../images/desktop_avatar.png} 
\caption{Selección del asistente. Versión de escritorio}
\label{fig:desktop avatar}
\end{figure}

\subsubsection{Chat}
Desde un principio se tiene claro que el diseño debe girar alrededor de la interacción con el \textit{bot}, y por tanto, el chat debe ser el elemento principal de la aplicación. Por ello, en la versión móvil se opta por una disposición donde prácticamente todo el espacio es ocupado por la conversación (ver Figura \ref{fig:mobile chat}). En cambio, en la versión de escritorio el chat ocupa la mitad derecha de la pantalla, mostrando en la parte izquierda la imagen del avatar escogido con un pequeño texto informativo (ver Figura \ref{fig:desktop chat}). El estilo del chat sigue los patrones de diseño típicos y más o menos estandarizados de aplicaciones de mensajería instantánea como \textit{WhatsApp} o \textit{Telegram}. El motivo detrás de esta decisión es facilitar la usabilidad, ya que prácticamente la totalidad de los usuarios están familiarizados con este tipo de interfaces.\\

Tal y como se puede observar en las Figuras \ref{fig:mobile chat} y \ref{fig:desktop chat}, en la cabecera superior encontramos los elementos habituales: la imagen de perfil de con quién se está hablando y un pequeño texto que nos indica el estado: \textit{en línea} o \textit{escribiendo}. Ambos elementos juegan una papel importante en la simulación: mientras la conversación transcurre el usuario ve constantemente el avatar del asistente e inconscientemente realiza una asociación entre los mensajes y la imagen de la persona que esta viendo. Por otra parte, aunque el usuario sepa que es una simulación observar como el chatbot va cambiando su estado según esté escribiendo o no, vuelve a reforzar la idea debido al paralelismo con las aplicaciones anteriormente mencionadas.\\

Por último, a través del icono de la esquina superior derecha se accede a consultar la información adicional.\\

\begin{figure}[htbp]
\centering
\includegraphics[scale=0.2]{../images/mobile_chat.png} 
\caption{Chat. Versión móvil}
\label{fig:mobile chat}
\end{figure}

\begin{figure}[htbp]
\centering
\includegraphics[scale=0.2]{../images/desktop_chat.png} 
\caption{Chat. Versión de escritorio}
\label{fig:desktop chat}
\end{figure}


\subsubsection{Consulta de información}
La pantalla de información contiene una breve introducción al VIH y una serie de temas anidados que se pueden ir desplegando y consultando. Además, en la parte superior existe una barra de búsqueda que permite encontrar rápidamente ciertos términos o palabras clave. Si existen resultados que coincidan con la búsqueda se indica con un pequeño indicador amarillo sobre el desplegable y resaltando el término en cuestión (ver Figura \ref{fig:mobile search} y \ref{fig:desktop search}).\\

\begin{figure}[htbp]
\centering
\includegraphics[scale=0.2]{../images/mobile_search.png} 
\caption{Búsqueda de información. Versión móvil}
\label{fig:mobile search}
\end{figure}

\begin{figure}[htbp]
\centering
\includegraphics[scale=0.2]{../images/desktop_search.png} 
\caption{Búsqueda de información. Versión de escritorio}
\label{fig:desktop search}
\end{figure}

Estos \textit{mockups} se envían con un pequeño vídeo y encuesta a los colaboradores la Unidad de Enfermedades Infecciosas del Hospital General de Elche para obtener su \textit{feedback} y aprobación. Tanto sus indicaciones como la versión interactiva pueden encontrarse en los documentos anexos a este trabajo.\\

\section{Implementación}
Una vez descrita la propuesta técnica y visualmente, se procede a detallar los distintos pasos seguidos para llevar a cabo la implementación del servicio.\\

\subsection{Conjunto de entrenamiento}
El conjunto de entrenamiento o \textit{dataset} hace referencia a los datos que se utilizan para entrenar un sistema conversacional en la tarea de reconocer las intenciones expresadas por los usuarios y en responder adecuadamente. En \textit{Rasa}, estos datos se estructuran mediante una serie de archivos en formato \textit{YAML} (\textit{YAML Ain't Markup Language}) que contienen las distintas entidades que representan una conversación. Así, los \textit{intents}  (intenciones) definen las diferentes preguntas que los usuarios pueden realizar, las respuestas contienen los mensajes que el chatbot devolverá y las \textit{stories} (historias) establecen los patrones de conversación. Con todos estos datos, posteriormente el asistente trata de identificar correctamente las intenciones expresadas por los usuarios y responder en consecuencia, según le haya sido indicado.\\

Para que el asistente pueda mantener conversaciones lo más amplias y variadas posibles, se han añadido al conjunto de entrenamiento \textit{intents}, respuestas y \textit{stories} pertenecientes a 3 áreas distintas: conocimiento médico, expresiones básicas y cháchara. Los datos médicos están orientados a responder todas las dudas médicas sobre el VIH y han sido extraídos de la hoja de cálculo facilitada por parte de la Unidad de Enfermedades Infecciosas del Hospital General de Elche. Este documento consta de 130 preguntas con sus correspondientes respuestas y contiene todo el conocimiento médico que inicialmente tendrá el chatbot.\\

A parte de todas las cuestiones relacionadas con el VIH, el asistente debe ser también capaz de entender y desenvolverse con una serie de expresiones básicas que suelen formar parte un diálogo. Las conversaciones suelen empezar con un \textit{Hola} o \textit{¡Buenos días!} y suelen ser correspondidas con otro saludo, seguidas de un \textit{¿Cómo estás?} o \textit{¿Cómo va?} y sus correspondientes respuestas. Durante el transcurso, también pueden aparecer otras expresiones como afirmaciones o negaciones (\textit{¡Por supuesto!}, \textit{Claro}, \textit{Para nada} etc) o peticiones como \textit{¿Me puedes ayudar?} o \textit{Necesito tu ayuda}. Finalmente, se suele usar un \textit{Adiós} o \textit{Hasta luego} para despedirse y finalizar la conversación. Todas estas expresiones y muchas otras aparecen tarde o temprano en una conversación, y por tanto, se han incorporado el mayor numero posible de ellas al conjunto de entrenamiento para que el asistente las entienda y pueda hacer uso de ellas. En la Figura 4.8 se muestra un ejemplo de una pequeña conversación donde se utilizan varias de estas expresiones.\\

\begin{figure}[htbp]
\centering
\includegraphics[scale=0.15]{../images/basics_chat.png}
\includegraphics[scale=0.15]{../images/basics_chat_2.png}
\caption{Ejemplo de una pequeña conversación}
\label{fig:expresiones auxiliares}
\end{figure}

Además, también es común que los usuarios intenten mantener un poco de cháchara con el bot. Algunos de los ejemplos más repetidos como \textit{¿Cómo te llamas?}, \textit{Quién te ha creado?} o \textit{¿Qué tiempo hace?} han sido también incorporados al conjunto. A continuación se detalla para cada entidad el proceso que se ha seguido para incorporar la información al \textit{dataset}.\\

\subsubsection{\textit{Intents}}
Cada vez que escribimos un mensaje expresamos una intención. Los \textit{intents} representan y definen estas intenciones recopilando una gran cantidad de ejemplos reales. Es importante tener en cuenta que una determinada intención puede ser expresada de distintas maneras. Pongamos por caso la pregunta  \textit{¿Existe cura para el VIH?} extraída de la hoja de cálculo. La cuestión expresa la intención del usuario de querer saber si existe algún tratamiento que cure completamente el VIH. Sin embargo, esta es simplemente una  de las múltiples formas que existen de preguntarlo, pues sería igualmente posible formular \textit{¿Hay cura para el VIH?} o \textit{¿El VIH tiene cura?} expresando la misma intención.\\

Por tanto, un usuario puede expresar la misma intención a través de diferentes expresiones. Por ello, cada pregunta o expresión a la cuál que se quiera poder responder, se debe analizar para identificar la intención, y en base a esta, elaborar una serie de variaciones que garanticen un amplio espectro de reconocimiento.
Estas variaciones consisten en:

\begin{itemize}
	\item Reformulaciones parciales o completas de las preguntas.
	\item Variantes con faltas de ortografía.
	\item Variantes con errores de escritura.
\end{itemize}

En consecuencia, y siguiendo con el ejemplo anterior, para \textit{¿Existe cura para el VIH?} se pueden obtener diferentes variaciones como las siguientes:

\begin{verbatim}
- intent: cura_vih
  examples: |
    - ¿Existe cura para el VIH?
    - existe ya cura para el vih?
    - han inventado alguna cura para el vih?
    - que cura hay para el vih?
    - se puede sanar el vih?
    - es posible expulsar el virus del cuerpo?
    - es posible exulsar el vih?
    - tiene cura el sida?
    - como se cura elsida?
    - puede alguienr ecuperarse completamente del vih?
    - ...
\end{verbatim}

Es importante considerar que, para que posteriormente el modelo generado distinga de forma fiable una intención de otra, los ejemplos deben ser distintos entre los \textit{intents}. Es decir, no se debe utilizar el mismo ejemplo de entrenamiento para dos intenciones diferentes. Si los ejemplos de entrenamiento resultan demasiado similares, se produce una confusión de intenciones \cite{bestPracticesNLU}.\\ 

En esta fase inicial de la implementación, y siguiendo las recomendaciones de \textit{Rasa}, se generan manualmente 30 variaciones por cada intención. En total, se implementan 167 \textit{intents}, (133 acerca del VIH, 23 expresiones básicas y 11 de cháchara). Con esto se pretende garantizar unos resultados suficientemente satisfactorios para lanzar un primer prototipo. En las siguientes fases del desarrollo el número de variaciones aumentará con la recogida de datos reales (ver sección \ref{cdd}).\\

\subsubsection{Respuestas}
Las respuestas ofrecidas por el asistente son la pieza clave para formar y transmitir personalidad. En Vihrtual-App se ha querido crear y dotar al \textit{bot} de una personalidad propia implementando los rasgos comentados en la sección \ref{interaccion} \textit{Interacción y características sociales}.\\

Como ejemplo, a continuación se puede observar las posibles respuestas que el chatbot puede dar a un usuario que le pregunte \textit{¿Cómo estas?}. Al responder el chatbot escoge al azar una de las variantes, añadiendo cierto dinamismo a las contestaciones para intentar no transmitir la sensación robótica de los mensajes predefinidos. Además, los textos han sido construidas utilizando varios de los recursos mencionados anteriormente, como el uso de buenos modales, el tono algo informal y la proactividad. \\

\begin{verbatim}
  utter_estado:
    - text: Genial! ¿Tú que tal?
    - text: Genial, la verdad. Muchas gracias por preguntar.
    - text: Hoy me encuentro genial, ¡gracias por preguntar!
    - text: Hoy me siento especialmente bien. ¿Y tú que tal?
    - ...
\end{verbatim}

En el caso de las información médica, se detecta que las respuestas proporcionadas por el hospital son, en muchos casos, difíciles de entender por parte de los usuarios. Algunas utilizan un lenguaje muy técnico y otras un registro demasiado formal que dificulta que el usuario establezca un vínculo con el \textit{bot}. Además, en muchos casos la longitud del mensaje es excesiva, pudiendo provocar rechazo y saturación \cite{shouldInteract}. En consecuencia, y por la propia naturaleza del medio, es recomendable sintetizar y aportar sólo la información relevante.\\

Por tanto, es necesario aplicar distintas técnicas para dinamizar las respuestas y no aborrecer al usuario. La información debe ser concreta, responder directamente a la pregunta y en la medida de lo posible intentar captar su intención. Esta tarea se realiza mediante la introducción de modificaciones tales como:

\begin{itemize}
	\item División de las respuestas más largas en varios mensajes.
	\item Ligeros cambios para variar el registro por otro menos formal.
	\item Supresión de afirmaciones y negaciones rotundas para adaptar las respuestas a una mayor variedad de preguntas.
	\item Hacer uso de algunos emoticonos para amenizar a ciertas expresiones o datos.
\end{itemize}

Dado el carácter informativo del servicio estas alteraciones se introducen intentando encontrar un equilibrio entre la calidad de la información y la síntesis. Como ejemplo, en la figura \ref{fig:modified response} se puede observar la respuesta original a la pregunta \textit{¿Qué es la Infección aguda por el VIH?} y su versión final tras las modificaciones.

\begin{figure}[htbp]
\centering
\includegraphics[scale=0.3]{../images/original_response.png}
\includegraphics[scale=0.3]{../images/modified_response.png}
\caption{Ejemplo de transformación de una respuesta}
\label{fig:modified response}
\end{figure}

\subsubsection{\textit{Stories}}
Una vez el conjunto de entrenamiento cuenta con la información de los \textit{intents} para identificar las expresiones y sus correspondientes respuestas, es necesario indicarle como debe utilizarlas. Para ello, se incorpora también a los datos de entrenamiento estructuras o patrones de conversación denominadas \textit{stories}. Las historias o \textit{stories} son representaciones de conversaciones entre usuarios y el chatbot. Estos datos utilizan un formato específico en el que la información introducida por el usuario se expresa como \textit{intents} y las respuestas como acciones del asistente \cite{rasaStories}. Por tanto, los \textit{intents} permiten identificar los mensajes de entrada, las respuestas contienen la información, y las \textit{stories} aportan el contexto y la lógica necesaria para responder coherentemente.\\

\begin{verbatim}
  - story: Pregunta por la cura + casos que se han curado + paciente berlin
    steps:
      - intent: cura_vih
      - action: utter_cura_vih
      - intent: faq/casos_cura
      - action: utter_faq
      - intent: faq/paciente_berlin
      - action: utter_faq

  - story: Agradecer + necesita algo más
    steps:
      - intent: agradecer
      - action: utter_no_hay_de_que
      - action: utter_algo_mas
      - intent: afirmar
      - action: utter_ofrecer_ayuda
\end{verbatim}


\subsubsection{Control de daños}
Todo asistente conversacional, por muy bien diseñado que esté, se enfrentará tarde o temprano a una situación no contemplada previamente. El control de daños (\textit{damage control}) son las diferentes técnicas que se pueden implementar para tratar de recuperarse en situaciones conflictivas o donde el chatbot falle \cite{shouldInteract}.\\% A continuación se detalla las diferentes estrategias implementadas en \textit{Vihrtual-App}.\\

%\paragraph{Falta de conocimiento}
Durante el devenir de una conversación es bastante probable que un usuario realice una pregunta que esté fuera del ámbito de conocimiento del chatbot, bien sea por desconocimiento o por poner a prueba el \textit{bot}. Cuando un asistente no responde adecuadamente a este tipo de preguntas puede crear un sentimiento de decepción en el usuario o incluso alentarle a continuar con el comportamiento abusivo \cite{shouldInteract}.\\

Por tanto, todos los casos detectados durante la recogida de información se han recopilado y agrupado en varios \textit{intents} para poder identificar correctamente estas situaciones y ofrecer una respuesta adecuada, intentado siempre reconducir la situación. En la figura \ref{fig:out1} se puede observar como el asistente reconoce la intención del usuario como fuera del ámbito, le responde disculpándose e indicando que no puede ayudarle en esa tarea y le sugiere algunas preguntas a través de unos botones de respuesta rápida.\\

\begin{figure}[htbp]
\centering
\includegraphics[scale=0.15]{../images/out_of_scope.png}
\caption{Ejemplo de manejo de mensajes fuera de ámbito}
\label{fig:out1}
\end{figure}


Por otra arte, los chatbots suelen ser más objeto de insultos y vejaciones de lo que un humano en una conversación real seria \cite{shouldInteract}. En este caso, la estrategia es idéntica, se recogen los datos necesarios para reconocer las situaciones y se responde para evitar que el usuario continue con su comportamiento.\\


Por último, y como último recurso, cuando el chabot no es capaz de identificar con suficiente confianza un \textit{intent}, muestra un mensaje disculpándose y preguntando amablemente si el usuario es capaz de reformular la pregunta. De esta manera, se intenta forzar al usuario a expresar su intención en otras palabras para poder tener una nueva oportunidad de reconocimiento. En la figura \ref{fig:recover} se puede observar un caso como el expuesto donde \textit{bot} es capaz de recuperase ante un fallo inicial.\\

\begin{figure}[htbp]
\centering
\includegraphics[scale=0.15]{../images/out_of_scope_2.png}
\includegraphics[scale=0.15]{../images/recover.png}
\caption{Ejemplo de recuperación ante insultos y identificación de \textit{intent}}
\label{fig:recover}
\end{figure}

Tras procesar toda esta información y realizar el entrenamiento, \textit{Rasa} genera un modelo con el cual el asistente es capaz de analizar las preguntas realizadas por los usuarios, asociarlas a uno de los \textit{intents} que ya tenga definidos (en base a una probabilidad) y ofrecer la respuesta correspondiente según le haya sido indicado.\\

\subsection{Servidor}
Una vez el entrenamiento ha finalizado el siguiente paso es hacer accesible el modelo a través de una \textit{API}. Para ello \textit{Rasa} pone a disposición un servidor ya integrado dentro del propio \textit{framework} con el que se puede publicar el modelo. Simplemente es necesario configurar los puertos por los cuáles se consumirá el servicio y ejecutar los procesos.

\subsection{Aplicación web}
Teniendo en cuenta el carácter del servicio, se decide que el chatbot se ponga a disposición del público a través de una web de libre acceso y sin requerir registro. El objetivo es facilitar a los usuarios el acceso para que cualquier persona pueda acceder al servicio sin necesidad de introducir ningún dato personal ni instalar ninguna aplicación en su terminal.\\

Debido a su sencillez, la aplicación se ha construido sin hacer uso de ningún \textit{framework} web, simplemente utilizando \textit{HTML}, \textit{CSS} y \textit{Javascript}. La interfaz sigue los diseños expuestos en la sección \ref{mockups}, implementando un diseño \textit{responsive}, hecho con \textit{flexbox} y que se adapta a distintos dispositivos.\\ 

Por otro lado, la mayor parte de lógica de la aplicación ha sido implementada utilizando una versión modificada de la librería \textit{Chatbot-Widget}. Con ella se realiza la comunicación entre el servidor y la web, enviando y recibiendo todos los mensajes.

\begin{figure}[htbp]
\centering
\includegraphics[scale=0.10]{../images/app.png}
\includegraphics[scale=0.10]{../images/app_2.png}
\includegraphics[scale=0.10]{../images/app_3.png}
\caption{Diferentes vistas de la \textit{app}}
\label{fig:vistas}
\end{figure}


\subsection{Despliegue}
Una vez implementados el servidor y el \textit{frontend} web es necesario publicar una versión de prueba del servicio para poder empezar con la fase de obtención de datos de conversaciones reales. Para ello, la Universitat Politècnica de València pone a disposición del proyecto una maquina virtual con \textit{Ubuntu 18.04} donde poder realizar el despliegue. Mediante acceso \textit{SSH} se instalan todas las librerías necesarias (\textit{Apache}, \textit{Python}, \textit{Rasa}, \textit{Spacy} etc) y se ponen en marcha tanto la web como el servidor.\\

Una vez el sistema está funcionando, se realiza una pequeña prueba con \textit{Rasa X} para asegurarse que la recogida de datos funciona correctamente. Tras ello, se solicita a la UPV la apertura de los puertos 80, 5055 y 5005 para permitir el tráfico \textit{HTTP} (ver figura \ref{fig:ports}), y así, permitir que usuarios externos a la universidad puedan acceder al servicio. De esta manera, el servicio se encuentra disponible desde \textit{http://vihrtualapp.gti-ia.upv.es} y se puede empezar con la recogida de datos.\\

\begin{figure}[htbp]
\centering
\includegraphics[scale=0.3]{../images/ports.png} 
\caption{Diagrama redirección de puertos y servicios} 
\label{fig:ports}
\end{figure}


\subsection{Desarrollo basado en conversaciones}
\label{cdd}
Uno de los mayores problemas que se suelen presentar durante el desarrollo de un chatbot es que inicialmente, el \textit{dataset} suele estar creado por los desarrolladores, y por tanto, sesgado y alejado de situaciones reales. El desarrollo basado en conversaciones (\textit{Conversation-Driven Development}) es el proceso de escuchar a los usuarios y utilizar esa información para mejorar el asistente \cite{conversationDriven}. Hacer el chatbot robusto puede ser un auténtico reto pues los usuarios siempre introducirán datos que inicialmente no estaban contemplados en el conjunto de entrenamiento. En consecuencia, este enfoque es la mejor manera de solucionar esta problemática, pues toda nueva información se incorpora al \textit{dataset} y así el modelo se nutre de situaciones y expresiones reales. Esta estrategia está especialmente recomendada por los desarrolladores de \textit{Rasa} \cite{bestPracticesNLU}.\\

Tras publicar la primera versión de prueba del chatbot el proyecto se encuentra listo para entrar en esta fase de desarrollo centrada en la recogida de datos. Para ello, durante varias semanas se comparte el chatbot, se revisan las conversaciones obtenidas, y se anotan, clasifican e incorporaran al \textit{dataset} todos los datos recopilados. Tras realizar todos estos cambios sobre el conjunto de entrenamiento se ejecutan los \textit{tests} correspondientes, se corrigen posibles errores y se vuelve a compartir el chatbot para una nueva iteración (ver figura \ref{cdd}).\\

\begin{figure}[htbp]
\centering
\includegraphics[scale=0.15]{../images/cdd.png} 
\caption{Ciclo del desarrollo basado en conversaciones}
\label{fig:cdd}
\end{figure}

\begin{figure}[htbp]
\centering
\includegraphics[scale=0.3]{../images/collected_dialogs.png} 
\caption{Ejemplo de una de las conversaciones recogidas}
\label{fig:collected dialogs}
\end{figure}

Para esta nueva fase del desarrollo se hizo uso del servicio \textit{Rasa X}, el cuál facilitaba la consulta y corrección de todas las conversaciones recogidas. Durante todo este proceso se recogieron y analizaron aproximadamente 60 conversaciones de las cuáles se pudo extraer e incorporar a la base de conocimiento la siguiente información:\\

\begin{itemize}
	\item 40 preguntas nuevas (con sus correspondientes respuestas) que inicialmente no estaban previstas.
	\item 232 variantes o expresiones.
\end{itemize}


\begin{figure}[htbp]
\centering
\includegraphics[scale=0.3]{../images/collected_nlu.png} 
\caption{Datos NLU recopilados y su etiquetado}
\label{fig:collected nlu}
\end{figure}

\subsection{Tests}
Tal y como se ha expuesto anteriormente en la sección \ref{cdd}, durante el desarrollo basado en conversaciones una parte importante del ciclo cosiste en realizar frecuentemente \textit{tests} automáticos. Tradicionalmente, en el mundo del \textit{software} se entiende por \textit{test} una prueba que verifica que una función devuelve el valor esperado. En \textit{Rasa}, los \textit{test} miden si los modelos generados hacen las predicciones correctas \cite{rasaTests}.\\

\subsection{Distribución}
Adicionalmente, y una vez el chatbot alcance la madurez necesaria, se publicará para aquellos usuarios que lo deseen una aplicación web empaquetada tanto para \textit{Android} como para \textit{iOS} utilizando el framework \textit{Capacitor}. Esto permitirá que aquellos usuarios interesados puedan instalar el servicio desde las tiendas de aplicaciones habituales en sus dispositivos móviles.\\




\chapter{Pruebas de validación}
\section{Tests}
test stories

\section{Evaluación del modelo NLU}
split nlu 

\section{Valoración médica}

\section{Valoración de los usuarios}



\chapter{Conclusión}
Durante la realización de este trabajo se han alcanzado con éxitos los objetivos planteados inicialmente. Se ha explorado el estado actual de creación de chatbots y se han estudiado distintos principios de diseño para aplicarlos al proyecto. Posteriormente, todo este conocimiento adquirido se ha plasmado en el diseño e implementación del asistente virtual, implementando todas las funcionalidades inicialmente previstas.\\

Tras su publicación, se han podido recopilar las conversaciones mantenidas por los usuarios con el servicio, de manera que ha sido posible trabajar en base a esta información para ir corrigiendo y mejorando el servicio. Este modo de trabajo ha producido una buena sinergia con los colaboradores de la Unidad de Enfermedades Infecciosas del Hospital General de Elche, que en última instancia, ha repercutido positivamente en el desarrollo. Gracias a la recogida de datos y su colaboración, ha sido posible recopilar y responder hasta 40 nuevas preguntas que inicialmente no estaban previstas, añadiendo una notable mejora a la base de conocimiento del chatbot.\\

Por otra parte, también se ha realizado una evaluación de usabilidad con 12 participantes que respondieron a los cuestionarios validados SUS y CUQ. Ambas pruebas arrojaron unos resultados de 85.625 y 80.342 sobre 100 respectivamente. Estas valoraciones positivas corroboran, en gran parte, muchas de las características sociales implementadas en el diseño de interacción. Por otro lado, también han servido para mostrar aquellos puntos donde es necesario continuar trabajando para mejorar la experiencia. En especial, el reconocimiento de entradas tiene margen de mejora y la cantidad de preguntas que puede responder el chatbot se debe aumentar.\\

Es importante considerar que, sólo 12 usuarios participaron en la prueba, y de ellos, la mayor parte fueron estudiantes. Por tanto, el análisis de los resultados debe realizarse teniendo en cuenta este contexto. Sería necesario un estudio más extenso donde, tanto la cantidad de encuestados como la variedad de sus perfiles fuera mayor. En este caso, gran parte de los participantes eran jóvenes y tenían estudios tecnológicos, por lo que los resultados obtenidos pueden estar algo sesgados.\\

En general, los resultados obtenidos han sido satisfactorios, y se espera que continuando con desarrollo basado en conversaciones se corrijan todos aquellos aspectos que son susceptibles de mejorar. En un futuro, la intención es que el chatbot sea también capaz de responder a otros temas relacionados con el VIH como puedan ser las las enfermedades de transmisión sexual.\\

A nivel personal, este trabajo de final de grado ha supuesto todo un impulso a la formación recibida en el grado. El reto de diseñar, dirigir e implementar una idea ha sido toda una experiencia. Durante todo el desarrollo ha sido necesario un aprendizaje permanente para poder resolver los problemas inicialmente planteados, superar las dificultades y lidiar con la planificación del proyecto. Igualmente, trabajar con un grupo de profesionales de un ámbito distinto ha supuesto una mejora de mis capacidades comunicativas, y sobretodo, una experiencia muy enriquecedora. Este trabajo ha supuesto para mi toda una introducción en el mundo del desarrollo de chatbots, y llevar a cabo un proyecto como este me ha otorgado las aptitudes necesarias para orientar mi futura vida profesional hacia el sector de la inteligencia artificial.\\


\appendix
\chapter{Apéndice}
\input{chapters/apéndice}

\end{document}