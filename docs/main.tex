\documentclass[11pt,a4paper]{book}
\usepackage[utf8]{inputenc}
\usepackage{float}
\usepackage{tikz}
\usepackage{graphicx}
\graphicspath{ {./images/} }
\usepackage{fancyhdr}
\usepackage[spanish,es-tabla]{babel}
\usepackage[official]{eurosym}
%\usepackage{csquotes}
\usepackage[left=2cm,right=2cm,top=2cm,bottom=2cm]{geometry}
\usepackage[backend=bibtex,sorting=none]{biblatex}
\addbibresource{bibliografia.bib}



\title{
{Borrador TFG Vihrtual-App}\\
{\large Universitat Politècnica de València}\\
%{\includegraphics{university.jpg}}
}
\author{Joan Ciprià Moreno Teodoro}
\date{Gandía, 2021}

\fancyhf{}
\cfoot{\thepage}
\pagestyle{fancy}

\begin{document}

\maketitle

\chapter*{Resumen}
La inteligencia artificial (IA) está transformando el mundo actual y la manera en la cual nos comunicamos con las máquinas. El lenguaje natural se está revelando como una forma muy óptima y competente de crear interfaces personalizadas que permitan a cada usuario interactuar utilizando sus propias palabras. Ante este nuevo escenario empezamos a ver como se popularizan los chatbots: programas diseñados para interactuar con los usuarios utilizando lenguaje natural. En el ámbito sanitario esto no ha sido una excepción y se siguen estos avances con gran interés en busca de asistentes virtuales que permitan mejorar y automatizar ciertos procesos médicos. En España, por ejemplo, en el año 2019 se notificaron 2.698 nuevos diagnósticos de VIH, de los cuales el 45.9\% presentaron diagnóstico tardío. Bajo este contexto surge VIHrtual-App, un proyecto que busca ayudar a detener la transmisión e informar ofreciendo un chatbot al público donde puedan obtener información veraz y relevante. En este documento se describe todo el proceso de estudio previo, diseño e implentación. Finalmente, se realiza una validación de usabilidad con usuarios reales, obteniendo unos resultados satisfactorios.\\
\\
\textbf{Palabras clave:} chatbots, virus de la inmunodeficiencia humana, lenguaje natural, inteligencia artificial, salud.

\chapter*{Abstract}
Artificial intelligence (AI) is transforming today's world and the way we communicate with machines. Natural language is proving to be very optimal to create customized interfaces which allow each user to interact with machines using their own words. In this new scenario, chatbots (programs designed to interact with users using natural language are growing in popularity. In health scope these advances are gathering interest interest to search virtual assistants to improve and automate certain medical processes. On the other hand, in 2019 2.698 new HIV diagnoses were reported in Spain, 45.9\% of them presented late diagnosis. With this context, VIHrtual-App is a project that seeks to help stop transmission and inform by offering a chatbot to the public where they can obtain truthful and relevant information. This document describes the whole process of preliminary study, design and implementation. Finally, a usability validation is carried out with real users, obtaining satisfactory results.\\
\\
\textbf{Keywords:} chatbots, human immunodeficiency virus, natural language, artificial intelligence, health. 

\tableofcontents

{\listoffigures \let\cleardoublepage\clearpage \listoftables}


\chapter{Introducción}
\input{chapters/introducción}

\chapter{Metodología}
\input{chapters/metodología}

\chapter{Marco Teórico}
\input{chapters/marco_teórico}

\chapter{Desarrollo de la Propuesta}
 


%\section{Solución propuesta}
%\section{Arquitectura de la aplicación}
%Después de analizar detenidamente los requisitos se procede a detallar la solución propuesta para el desarrollo de Vihrtual-App.


\subsection{Diseño de software}
[TODO: Diagramas del diseño de software]


Este propuesta se presenta en una reunión con los colaboradores la Unidad de Enfermedades Infecciosas del Hospital General de Elche para obtener su aprobación. Tanto las notas de la reunión como las diapositivas se pueden encontrar en los documentos anexos a este trabajo.\\


\section{Diseño}
\input{chapters/desarrollo_propuesta/diseño}

\section{Implementación}
\input{chapters/desarrollo_propuesta/implementación}

\chapter{Pruebas de validación}
En esta sección se muestran las distintas pruebas de validación\\
 

\section{Evaluación del modelo NLU}
split nlu 

\section{Evaluación de usabilidad}
Finalmente, se ha realizado una prueba de usabilidad a la aplicación para detectar posibles problemas y corregirlos. Los participantes fueron captados a través de una invitación por email enviada por los tutores del TFG. El correo incluía un breve texto indicando a los participantes que probaran durante unos minutos el chatbot y que posteriormente realizaran un cuestionario a través de un link. Todos los participantes que realizaron el cuestionario han sido incluidos en los resultados. En total, 10 adultos de entre 16 y 60 años participaron en la prueba durante un periodo de 7 días.\\

Para este estudio se han utilizado los cuestionarios validados \textit{System Usability Scale} (SUS) \cite{dirtySUS} y \textit{Chatbot Usability Questionnaire} (CUQ) \cite{cuq}. SUS es un cuestionario genérico diseñado para obtener una evaluación general y rápida sobre la usabildiad de una determinada aplicación \cite{dirtySUS}. Se compone de 10 sentencias sobre las cuales los usuarios deben valorar en una escala del 1 al 5 su total disconformidad (1) o total conformidad (5). Se utilizó una versión del SUS traducida al español y validada \cite{spanishSUS}.\\

Por otro lado, CUQ es un cuestionario de usabilidad específico que evalúa la personalidad, la inteligencia, el entendimiento, la navegación y el manejo de errores de un chatbot \cite{cuq}. CUQ está diseñado para ser comparable al SUS (utiliza la misma escala de valoración) pero utilizando 16 sentencias específicas para chatbots. Para su uso, se realizó una traducción lo más fiel posible al español.\\

\subsection{Cálculo de puntuaciones}
Se utilizó la hoja de cálculo de la puntuación de SUS \cite{susSpread} para calcular las puntuaciones del SUS sobre 100. La fórmula para este cálculo se muestra en la ecuación 1.\\


donde n=número de sujetos (cuestionarios), m=10 (número de preguntas), qi,j=puntuación individual por pregunta por participante, norma=2,5.\\



Las puntuaciones del CUQ se calcularon sobre 160 utilizando la fórmula de la ecuación 2, y luego se normalizaron para dar una puntuación sobre 100, permitiendo así la comparación con SUS. \\


donde m = 16 (número de preguntas) y n = puntuación de la pregunta individual por participante \cite{cuq}.\\

\subsection{Resultados}


\chapter{Conclusiones}
\input{chapters/conclusión}

\printbibliography[heading=bibnumbered,title=Referencias]

%\appendix
%\chapter{Apéndice}
%\input{chapters/apéndice}

\end{document}