\documentclass[11pt]{book}
\usepackage[utf8]{inputenc}
\usepackage{graphicx}
\graphicspath{ {images/} }
%\usepackage{fancyhdr}
%\pagestyle{fancy}
\usepackage[spanish]{babel}
\usepackage{geometry}
\usepackage{biblatex}
\addbibresource{bibliografia.bib}

%----------------------------------------------------------------------------------------
%	MARGIN SETTINGS
%----------------------------------------------------------------------------------------

\geometry{
	paper=a4paper, % Change to letterpaper for US letter
	inner=2.5cm, % Inner margin
	outer=3.8cm, % Outer margin
	bindingoffset=.5cm, % Binding offset
	top=1.5cm, % Top margin
	bottom=1.5cm, % Bottom margin
	%showframe, % Uncomment to show how the type block is set on the page
}


\title{
{TFG Vihrtual-App}\\
{\large Universitat Politècnica de València}\\
%{\includegraphics{university.jpg}}
}
\author{Joan Ciprià Moreno Teodoro}
\date{Gandía, 2021}

\begin{document}

\maketitle

\chapter*{Resumen}
\cite{healthAgents}
La inteligencia artificial (IA) está transformando el mundo actual y la manera en la cual nos comunicamos con las máquinas, popularizándose cada vez más los sistemas basados en voz. Esto es debido a que el lenguaje natural se está revelando como una forma muy óptima y competente para crear interfaces personalizadas que permitan a cada usuario interactuar con sus propias palabras. \\

Ante este nuevo escenario empezamos a ver como se popularizan los chatbots: programas diseñados para interactuar con los usuarios utilizando un lenguaje natural (por voz o texto) con el objetivo de hacer creer al usuario está hablando con una persona real. \\

En el ámbito sanitario esto no ha sido una excepción y se siguen estas tecnologías con gran interés en busca de sistemas que permitan automatizar ciertos procesos médicos. \\

Bajo este contexto surge Vihrtual-App, un chatbot creado con la colaboración del Hospital universitario de Elche y la Universitat Politència de València con la intención de concienciar y resolver a la población dudas sobre el Virus de la inmunodeficiencia humana (VIH) y otras enfermedades de transmisión sexual.
\\
\\
\textbf{Palabras clave:} Inteligencia artificial, lenguaje natural, chatbots, salud, virus de la inmunodeficiencia humana.

\chapter*{Abstract}
Artificial Inteligence (AI) is moving forward our world and changing how we   communicate with machines, growing popularity sistems based on voice.
\\
\\
\textbf{Keywords:} Artificial inteligence, natural language, chatbots, health, human immunodeficiency virus. 

\tableofcontents

\chapter{Introducción}
\input{chapters/introducción}

\chapter{Marco teórico}
\input{chapters/marco_teórico}

\chapter{Desarrollo de la Propuesta}

\section{Requisitos}
 


\section{Metodología}
\input{chapters/desarrollo_propuesta/metodología}

\section{Diseño}
\input{chapters/desarrollo_propuesta/diseño}

\section{Implementación}
\input{chapters/desarrollo_propuesta/implementación}

\chapter{Pruebas de validación}
En esta sección se muestran las distintas pruebas de validación\\
 

\section{Evaluación del modelo NLU}
split nlu 

\section{Evaluación de usabilidad}
Finalmente, se ha realizado una prueba de usabilidad a la aplicación para detectar posibles problemas y corregirlos. Los participantes fueron captados a través de una invitación por email enviada por los tutores del TFG. El correo incluía un breve texto indicando a los participantes que probaran durante unos minutos el chatbot y que posteriormente realizaran un cuestionario a través de un link. Todos los participantes que realizaron el cuestionario han sido incluidos en los resultados. En total, 10 adultos de entre 16 y 60 años participaron en la prueba durante un periodo de 7 días.\\

Para este estudio se han utilizado los cuestionarios validados \textit{System Usability Scale} (SUS) \cite{dirtySUS} y \textit{Chatbot Usability Questionnaire} (CUQ) \cite{cuq}. SUS es un cuestionario genérico diseñado para obtener una evaluación general y rápida sobre la usabildiad de una determinada aplicación \cite{dirtySUS}. Se compone de 10 sentencias sobre las cuales los usuarios deben valorar en una escala del 1 al 5 su total disconformidad (1) o total conformidad (5). Se utilizó una versión del SUS traducida al español y validada \cite{spanishSUS}.\\

Por otro lado, CUQ es un cuestionario de usabilidad específico que evalúa la personalidad, la inteligencia, el entendimiento, la navegación y el manejo de errores de un chatbot \cite{cuq}. CUQ está diseñado para ser comparable al SUS (utiliza la misma escala de valoración) pero utilizando 16 sentencias específicas para chatbots. Para su uso, se realizó una traducción lo más fiel posible al español.\\

\subsection{Cálculo de puntuaciones}
Se utilizó la hoja de cálculo de la puntuación de SUS \cite{susSpread} para calcular las puntuaciones del SUS sobre 100. La fórmula para este cálculo se muestra en la ecuación 1.\\


donde n=número de sujetos (cuestionarios), m=10 (número de preguntas), qi,j=puntuación individual por pregunta por participante, norma=2,5.\\



Las puntuaciones del CUQ se calcularon sobre 160 utilizando la fórmula de la ecuación 2, y luego se normalizaron para dar una puntuación sobre 100, permitiendo así la comparación con SUS. \\


donde m = 16 (número de preguntas) y n = puntuación de la pregunta individual por participante \cite{cuq}.\\

\subsection{Resultados}


\chapter{Conclusiones}
\input{chapters/conclusión}

\printbibliography

\appendix
\chapter{Apéndice}
\input{chapters/apéndice}

\end{document}